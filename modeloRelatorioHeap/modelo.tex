\documentclass[12pt,a4paper,twoside]{report}
% -------------------------------------------------------------------- %
% Pacotes

\usepackage[utf8]{inputenc}
\usepackage[T1]{fontenc}
\usepackage[brazil]{babel}
\usepackage[fixlanguage]{babelbib}
\usepackage[pdftex]{graphicx}      % usamos arquivos pdf/png como figuras
\usepackage{setspace}              % espaçamento flexvel
\usepackage{indentfirst}           % indentação do primeiro parágrafo
\usepackage{makeidx}               % índice remissivo
\usepackage[nottoc]{tocbibind}     % acrescentamos a bibliografia/indice/conteudo no Table of Contents
\usepackage{courier}               % usa o Adobe Courier no lugar de Computer Modern Typewriter
\usepackage{type1cm}               % fontes realmente escaláveis
\usepackage{titletoc}
\usepackage{ucs}
\usepackage[font=small,format=plain,labelfont=bf,up,textfont=it,up]{caption}
\usepackage[usenames,svgnames,dvipsnames]{xcolor}
\usepackage[a4paper,top=2.54cm,bottom=2.0cm,left=2.0cm,right=2.54cm]{geometry} % margens
\usepackage{amsmath}
\usepackage{booktabs} % cria tabelas em formato profissional
\usepackage[pdftex,plainpages=false,pdfpagelabels,pagebackref,colorlinks=true,citecolor=DarkGreen,
linkcolor=NavyBlue,urlcolor=DarkRed,filecolor=green,bookmarksopen=true]{hyperref} % links coloridos
\usepackage[all]{hypcap}                % soluciona o problema com o hyperref e capítulos
\usepackage[square,sort,nonamebreak,comma]{natbib}  % citação bibliográfica alpha
\fontsize{60}{62}\usefont{OT1}{cmr}{m}{n}{\selectfont}
\usepackage{upquote}                    % formata apóstrofes '
\usepackage{textcomp}

% Para formatar corretamente as URLs
\usepackage{url}
% -------------------------------------------------------------------- %
% Cabeçalhos similares ao TAOCP de Donald E. Knuth
\usepackage{fancyhdr}
\pagestyle{fancy}
\fancyhf{}
\renewcommand{\chaptermark}[1]{\markboth{\MakeUppercase{#1}}{}}
\renewcommand{\sectionmark}[1]{\markright{\MakeUppercase{#1}}{}}
\renewcommand{\headrulewidth}{0pt}

\frenchspacing                     % arruma o espaço: id est (i.e.) e exempli gratia (e.g.)
\urlstyle{same}                    % URL com o mesmo estilo do texto e no mono-spaced
\makeindex                         % para o índice remissivo
\raggedbottom                      % para no permitir espaços extras no texto
\fontsize{60}{62}\usefont{OT1}{cmr}{m}{n}{\selectfont}
\cleardoublepage
\normalsize

% -------------------------------------------------------------------- %
% Cores para formatação de código
\usepackage{color}
\definecolor{vermelho}{rgb}{0.6,0,0} % para strings
\definecolor{verde}{rgb}{0.25,0.5,0.35} % para comentários
\definecolor{roxo}{rgb}{0.5,0,0.35} % para palavras-chaves
\definecolor{azul}{rgb}{0.25,0.35,0.75} % para strings
\definecolor{cinza-claro}{gray}{0.95}
% -------------------------------------------------------------------- %
% Opções de listagem usados para o código fonte
% Ref: http://en.wikibooks.org/wiki/LaTeX/Packages/Listings
\usepackage{listings}           % para formatar código-fonte (ex. em Java)

\lstset{ %
language=Python,                      % seleciona a linguagem do código
basicstyle=\footnotesize\ttfamily,    % o tamanho da fonte usado no código
commentstyle=\color{verde}\bfseries,  % formatação de comentários
stringstyle=\color{azul},             % formatação de strings
upquote=true,
numbers=left,                   % onde colocar os números de linha
numberstyle=\tiny,  % o tamanho da fonte usada para a numeração das linhas
stepnumber=1,                   % o intervalo entre dois números de linhas. Se for 1, numera cada uma.
numbersep=5pt,                  % how far the line-numbers are from the code
showspaces=false,               % show spaces adding particular underscores
showstringspaces=false,         % underline spaces within strings
showtabs=false,                 % show tabs within strings adding particular underscores
keywordstyle=\color{roxo}\bfseries,
keywordstyle=[1]\color{roxo}\bfseries,
keywordstyle=[2]\color{verde}\bfseries,
frame=b,                   % adds a frame around the code
framerule=0.6pt,
tabsize=2,                      % sets default tabsize to 2 spaces
captionpos=t,                   % sets the caption-position to top
breaklines=true,                % sets automatic line breaking
breakatwhitespace=false,        % sets if automatic breaks should only happen at whitespace
escapeinside={\%*}{*)},         % if you want to add a comment within your code
backgroundcolor=\color[rgb]{1.0,1.0,1.0}, % choose the background color.
rulecolor=\color[rgb]{0.8,0.8,0.8},
extendedchars=true,
xleftmargin=10pt,
xrightmargin=10pt,
framexleftmargin=10pt,
framexrightmargin=10pt,
literate={â}{{\^{a}}}1  % para formatar corretamente os acentos do Português ao usar utf8
    {ê}{{\^{e}}}1
    {ô}{{\^{o}}}1
    {Â}{{\^{A}}}1
    {Ê}{{\^{E}}}1
    {Ô}{{\^{O}}}1
    {á}{{\'{a}}}1
    {é}{{\'{e}}}1
    {í}{{\'{i}}}1
    {ó}{{\'{o}}}1
    {ú}{{\'{u}}}1
    {Á}{{\'{A}}}1
    {É}{{\'{E}}}1
    {Í}{{\'{I}}}1
    {Ó}{{\'{O}}}1
    {Ú}{{\'{U}}}1
    {à}{{\`{a}}}1
    {À}{{\`{A}}}1
    {ã}{{\~{a}}}1
    {õ}{{\~{o}}}1
    {Ã}{{\~{A}}}1
    {Õ}{{\~{O}}}1
    {ç}{{\c{c}}}1
    {Ç}{{\c{C}}}1
    {ü}{{\"u}}1
    {Ü}{{\"U}}1
}

\renewcommand{\lstlistingname}{Listagem}
\renewcommand{\lstlistlistingname}{Lista de Listagens}

% \captionsetup[lstlisting]{singlelinecheck=false, labelfont={blue}, textfont={blue}}
\usepackage{caption}
\DeclareCaptionFont{white}{\color{white}}
\DeclareCaptionFormat{listing}{\colorbox[cmyk]{0.43, 0.35, 0.35,0.01}{\parbox{\textwidth}{\hspace{15pt}#1#2#3}}}
\captionsetup[lstlisting]{format=listing,labelfont=white,textfont=white, singlelinecheck=false, margin=0pt, font={bf,footnotesize}}

\title{Análise experimental de algoritmos usando Python}
\author{Patrícia Mariana Ramos Marcolino \\
\texttt{\small \url{pmrmarcolino@hotmail.com}}
\vspace{1cm} \\
Eduardo Pinheiro Barbosa \\
\texttt{\small \url{eduardptu@hotmail.com}}
\vspace{1cm} \\
Faculdade de Computação \\
Universidade Federal de Uberlândia
}
\date{\today}

\begin{document}
\maketitle
% -------------------------------------------------------------------- %
% Listas de figuras, tabelas e códigos criadas automaticamente
\listoffigures
\listoftables
\lstlistoflistings
% -------------------------------------------------------------------- %

% -------------------------------------------------------------------- %
% Sumário
\tableofcontents
% cabeçalho para as páginas de todos os capítulos
\fancyhead[RE,LO]{\thesection}

%\singlespacing              % espaçamento simples
\setlength{\parskip}{0.15in} % espaçamento entre paragráfos

\chapter{Análise}

O algoritmo resolve o problema da ordenação usando um max-heap.  A rotina Constrói-Max-Heap transforma $1..n$ em um max-heap e a rotina Corrige-Descendo transforma o vetor $1..m−1$ em max-heap supondo que as subárvores com raizes 2 e 3 são max-heaps.

- o vetor $A[1..m]$ é um max-heap,
$A[1..m]  \leq  A[m+1..n]$  (ou seja, $A[1] \leq A[m+1]$, $A[2] \leq A[m+2]$, etc.)  e
- o vetor $A[m+1..n]$ está em ordem crescente.
- É claro que $A[1..n]$ estará em ordem crescente quando m for igual a 1.

No pior caso, Constrói-Max-Heap consome $Ο(n)$ unidades de tempo e cada invocação de Corrige-Descendo consome $Ο(\log(n)$ unidades de tempo.  Logo, Heapsort consome $Ο(n\log(n))$

unidades de tempo no pior caso. Assim, o algoritmo é linearítmico.

\chapter{Resultados}
\section{Tabelas}

\input{../heapsort/tabelas/heapsortAleatorio.tex}

\begin{figure}[ht]
\centering \includegraphics[scale=0.8]{../heapsort/imagens/heapsortAleatorio0.png}
\caption{A análise do grafico para $2^{32}$ segue abaixo para heapsort}

Tendo a função $T(n) = 5.012021\mathrm{e}-6*n\log_{2}(n)-0.0024072$ e para o $n =2^{32}$, $T(2^{32}) = 477472.3$
\label{fig:heapsortAleatorio0}
\end{figure}

\begin{figure}[ht]
\centering \includegraphics[scale=0.8]{../heapsort/imagens/heapsortAleatorio1.png}
\caption{A análise do grafico para $2^{32}$ segue abaixo para heapsort}

Tendo a função $T(n) = 0.11083\mathrm{e}-6*n\log_{2}(n)+138.4156$ e para o $n =2^{32}$, $T(2^{32}) = 10696.7$
\label{fig:heapsortAleatorio1}
\end{figure}


\input{../heapsort/tabelas/heapsortCrescente.tex}

\begin{figure}[ht]
\centering \includegraphics[scale=0.8]{../heapsort/imagens/heapsortCrescente0.png}
\caption{A análise do grafico para $2^{32}$ segue abaixo para heapsort}

Tendo a função $T(n) = 5.01858\mathrm{e}-6*n\log_{2}(n)-0.0009603$ e para o $n =2^{32}$, $T(2^{32}) = 4.78097*10^5$
\label{fig:heapsortCrescente0}
\end{figure}

\begin{figure}[ht]
\centering \includegraphics[scale=0.8]{../heapsort/imagens/heapsortCrescente1.png}
\caption{A análise do grafico para $2^{32}$ segue abaixo para heapsort}

Tendo a função $T(n) = 0.11083\mathrm{e}-6*n\log_{2}(n)+138.4156$ e para o $n =2^{32}$, $T(2^{32}) = 10696.7$
\label{fig:heapsortCrescente1}
\end{figure}


\input{../heapsort/tabelas/heapsortDecrescente.tex}

\begin{figure}[ht]
\centering \includegraphics[scale=0.8]{../heapsort/imagens/heapsortDecrescente0.png}
\caption{A análise do grafico para $2^{32}$ segue abaixo para heapsort}

Tendo a função $T(n) = 4.34285\mathrm{e}-6*n\log_{2}(n)+0.0009521$ e para o $n =2^{32}$, $T(2^{32}) = 4.13723*10^{5}$
\label{fig:heapsortDecrescente0}
\end{figure}

\begin{figure}[ht]
\centering \includegraphics[scale=0.8]{../heapsort/imagens/heapsortDecrescente1.png}
\caption{A análise do grafico para $2^{32}$ segue abaixo para heapsort}

Tendo a função $T(n) = 0.11083\mathrm{e}-6*n\log_{2}(n)+138.4156$ e para o $n =2^{32}$, $T(2^{32}) = 10696.7$
\label{fig:heapsortDecrescente1}
\end{figure}


\input{../heapsort/tabelas/heapsortQuaseCresc10.tex}

\begin{figure}[ht]
\centering \includegraphics[scale=0.8]{../heapsort/imagens/heapsortQuaseCresc100.png}
\caption{A análise do grafico para $2^{32}$ segue abaixo para heapsort}

Tendo a função $T(n) = 4.86048\mathrm{e}-6*n\log_{2}(n)+0.0007349$ e para o $n =2^{32}$, $T(2^{32}) = 4.63036*10^{5}$
\label{fig:heapsortQuaseCresc100}
\end{figure}

\begin{figure}[ht]
\centering \includegraphics[scale=0.8]{../heapsort/imagens/heapsortQuaseCresc101.png}
\caption{A análise do grafico para $2^{32}$ segue abaixo para heapsort}

Tendo a função $T(n) = 0.11083\mathrm{e}-6*n\log_{2}(n)+138.4156$ e para o $n =2^{32}$, $T(2^{32}) =10696.7 $
\label{fig:heapsortQuaseCresc101}
\end{figure}


\input{../heapsort/tabelas/heapsortQuaseCresc20.tex}

\begin{figure}[ht]
\centering \includegraphics[scale=0.8]{../heapsort/imagens/heapsortQuaseCresc200.png}
\caption{A análise do grafico para $2^{32}$ segue abaixo para heapsort}

Tendo a função $T(n) = 4.949708\mathrm{e}-6*n\log_{2}(n)-0.0006395$ e para o $n =2^{32}$, $T(2^{32}) = 471536.0$
\label{fig:heapsortQuaseCresc200}
\end{figure}

\begin{figure}[ht]
\centering \includegraphics[scale=0.8]{../heapsort/imagens/heapsortQuaseCresc201.png}
\caption{A análise do grafico para $2^{32}$ segue abaixo para heapsort}

Tendo a função $T(n) = 0.11083\mathrm{e}-6*n\log_{2}(n)+138.4156$ e para o $n =2^{32}$, $T(2^{32}) = 10696.7$
\label{fig:heapsortQuaseCresc201}
\end{figure}

\clearpage
\input{../heapsort/tabelas/heapsortQuaseCresc30.tex}

\begin{figure}[ht]
\centering \includegraphics[scale=0.8]{../heapsort/imagens/heapsortQuaseCresc300.png}
\caption{A análise do grafico para $2^{32}$ segue abaixo para heapsort}

Tendo a função $T(n) = 4.88631\mathrm{e}-6*n\log_{2}(n)+0.0006267$ e para o $n =2^{32}$, $T(2^{32}) = 0.169809$
\label{fig:heapsortQuaseCresc300}
\end{figure}

\begin{figure}[ht]
\centering \includegraphics[scale=0.8]{../heapsort/imagens/heapsortQuaseCresc301.png}
\caption{A análise do grafico para $2^{32}$ segue abaixo para heapsort}

Tendo a função $T(n) = 0.11083\mathrm{e}-6*n\log_{2}(n)+138.4156$ e para o $n =2^{32}$, $T(2^{32}) = 4.65496*10^{5}$
\label{fig:heapsortQuaseCresc301}
\end{figure}


\input{../heapsort/tabelas/heapsortQuaseCresc40.tex}

\begin{figure}[ht]
\centering \includegraphics[scale=0.8]{../heapsort/imagens/heapsortQuaseCresc400.png}
\caption{A análise do grafico para $2^{32}$ segue abaixo para heapsort}

Tendo a função $T(n) = 5.1525\mathrm{e}-6*n\log_{2}(n)-0.002120$ e para o $n =2^{32}$, $T(2^{32}) = 4.90855*10^5$
\label{fig:heapsortQuaseCresc400}
\end{figure}

\begin{figure}[ht]
\centering \includegraphics[scale=0.8]{../heapsort/imagens/heapsortQuaseCresc401.png}
\caption{A análise do grafico para $2^{32}$ segue abaixo para heapsort}

Tendo a função $T(n) = 0.11083\mathrm{e}-6*n\log_{2}(n)+138.4156$ e para o $n =2^{32}$, $T(2^{32}) = 10696.7$
\label{fig:heapsortQuaseCresc401}
\end{figure}


\input{../heapsort/tabelas/heapsortQuaseCresc50.tex}

\begin{figure}[ht]
\centering \includegraphics[scale=0.8]{../heapsort/imagens/heapsortQuaseCresc500.png}
\caption{A análise do grafico para $2^{32}$ segue abaixo para heapsort}

Tendo a função $T(n) = 4.7556\mathrm{e}-6*n\log_{2}(n)+0.00221$ e para o $n =2^{32}$, $T(2^{32}) = 4.53044*10^{5}$
\label{fig:heapsortQuaseCresc500}
\end{figure}

\begin{figure}[ht]
\centering \includegraphics[scale=0.8]{../heapsort/imagens/heapsortQuaseCresc501.png}
\caption{A análise do grafico para $2^{32}$ segue abaixo para heapsort}

Tendo a função $T(n) = 0.11083\mathrm{e}-6*n\log_{2}(n)+138.4156$ e para o $n =2^{32}$, $T(2^{32}) = 10696.7$
\label{fig:heapsortQuaseCresc501}
\end{figure}


\input{../heapsort/tabelas/heapsortQuaseDecresc10.tex}

\begin{figure}[ht]
\centering \includegraphics[scale=0.8]{../heapsort/imagens/heapsortQuaseDecresc100.png}
\caption{A análise do grafico para $2^{32}$ segue abaixo para heapsort}

Tendo a função $T(n) = 4.43692\mathrm{e}-6*n\log_{2}(n)-0.0004582$ e para o $n =2^{32}$, $T(2^{32}) = 4.22685*10^{5}$
\label{fig:heapsortQuaseDecresc100}
\end{figure}

\begin{figure}[ht]
\centering \includegraphics[scale=0.8]{../heapsort/imagens/heapsortQuaseDecresc101.png}
\caption{A análise do grafico para $2^{32}$ segue abaixo para heapsort}

Tendo a função $T(n) = 0.110836\mathrm{e}-6*n\log_{2}(n)+138.4156$ e para o $n =2^{32}$, $T(2^{32}) = 10697.3$
\label{fig:heapsortQuaseDecresc101}
\end{figure}


\input{../heapsort/tabelas/heapsortQuaseDecresc20.tex}

\begin{figure}[ht]
\centering \includegraphics[scale=0.8]{../heapsort/imagens/heapsortQuaseDecresc200.png}
\caption{A análise do grafico para $2^{32}$ segue abaixo para heapsort}

Tendo a função $T(n) = 4.43671\mathrm{e}-6*n\log_{2}(n)-0.00054047$ e para o $n =2^{32}$, $T(2^{32}) = 4.22665*10^{5}$
\label{fig:heapsortQuaseDecresc200}
\end{figure}

\begin{figure}[ht]
\centering \includegraphics[scale=0.8]{../heapsort/imagens/heapsortQuaseDecresc201.png}
\caption{A análise do grafico para $2^{32}$ segue abaixo para heapsort}

Tendo a função $T(n) = 0.110836\mathrm{e}-6*n\log_{2}(n)+138.4156$ e para o $n =2^{32}$, $T(2^{32}) = 10697.3$
\label{fig:heapsortQuaseDecresc201}
\end{figure}

\clearpage
\input{../heapsort/tabelas/heapsortQuaseDecresc30.tex}

\begin{figure}[ht]
\centering \includegraphics[scale=0.8]{../heapsort/imagens/heapsortQuaseDecresc300.png}
\caption{A análise do grafico para $2^{32}$ segue abaixo para heapsort}

Tendo a função $T(n) = 4.43941\mathrm{e}-6*n\log_{2}(n)+0.0003324$ e para o $n =2^{32}$, $T(2^{32}) = 4.22922*10^{5}$
\label{fig:heapsortQuaseDecresc300}
\end{figure}

\begin{figure}[ht]
\centering \includegraphics[scale=0.8]{../heapsort/imagens/heapsortQuaseDecresc301.png}
\caption{A análise do grafico para $2^{32}$ segue abaixo para heapsort}

Tendo a função $T(n) = 0.11083\mathrm{e}-6*n\log_{2}(n)+138.4156$ e para o $n =2^{32}$, $T(2^{32}) = 10696.7$
\label{fig:heapsortQuaseDecresc301}
\end{figure}


\input{../heapsort/tabelas/heapsortQuaseDecresc40.tex}

\begin{figure}[ht]
\centering \includegraphics[scale=0.8]{../heapsort/imagens/heapsortQuaseDecresc400.png}
\caption{A análise do grafico para $2^{32}$ segue abaixo para heapsort}

Tendo a função $T(n) = 4.63045\mathrm{e}-6*n\log_{2}(n)-0.0017405$ e para o $n =2^{32}$, $T(2^{32}) = 4.22922*10^{5}$
\label{fig:heapsortQuaseDecresc400}
\end{figure}

\begin{figure}[ht]
\centering \includegraphics[scale=0.8]{../heapsort/imagens/heapsortQuaseDecresc401.png}
\caption{A análise do grafico para $2^{32}$ segue abaixo para heapsort}

Tendo a função $T(n) = 0.11083\mathrm{e}-6*n\log_{2}(n)+138.41560$ e para o $n =2^{32}$, $T(2^{32}) = 10696.7$
\label{fig:heapsortQuaseDecresc401}
\end{figure}


\input{../heapsort/tabelas/heapsortQuaseDecresc50.tex}

\begin{figure}[ht]
\centering \includegraphics[scale=0.8]{../heapsort/imagens/heapsortQuaseDecresc500.png}
\caption{A análise do grafico para $2^{32}$ segue abaixo para heapsort}

Tendo a função $T(n) = 4.58151\mathrm{e}-6*n\log_{2}(n)-0.001291$ e para o $n =2^{32}$, $T(2^{32}) = 4.36459*10^{5}$
\label{fig:heapsortQuaseDecresc500}
\end{figure}

\begin{figure}[ht]
\centering \includegraphics[scale=0.8]{../heapsort/imagens/heapsortQuaseDecresc501.png}
\caption{A análise do grafico para $2^{32}$ segue abaixo para heapsort}

Tendo a função $T(n) = 0.11083\mathrm{e}-6*n\log_{2}(n)+138.4156$ e para o $n =2^{32}$, $T(2^{32}) = 10696.7$
\label{fig:heapsortQuaseDecresc501}
\end{figure}


\clearpage
\clearpage
\addcontentsline{toc}{part}{Apêndice}
\appendix

\chapter{Arquivo ../heapsort/heapsort.py \label{ap:heapsort}}
\lstinputlisting[caption={../heapsort/heapsort.py \label{arq:heapsort}}]{../heapsort/heapsort.py}

\chapter{Arquivo ../heapsort/ensaio.py \label{ap:heapsortensaio}}
\lstinputlisting[caption={../heapsort/ensaio.py \label{arq:heapsortensaio}}]{../heapsort/ensaio.py}

\end{document}
