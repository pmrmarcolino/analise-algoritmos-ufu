\documentclass[12pt,a4paper,twoside]{report}
% -------------------------------------------------------------------- %
% Pacotes

\usepackage[utf8]{inputenc}
\usepackage[T1]{fontenc}
\usepackage[brazil]{babel}
\usepackage[fixlanguage]{babelbib}
\usepackage[pdftex]{graphicx}      % usamos arquivos pdf/png como figuras
\usepackage{setspace}              % espaçamento flexvel
\usepackage{indentfirst}           % indentação do primeiro parágrafo
\usepackage{makeidx}               % índice remissivo
\usepackage[nottoc]{tocbibind}     % acrescentamos a bibliografia/indice/conteudo no Table of Contents
\usepackage{courier}               % usa o Adobe Courier no lugar de Computer Modern Typewriter
\usepackage{type1cm}               % fontes realmente escaláveis
\usepackage{titletoc}
\usepackage{ucs}
\usepackage[font=small,format=plain,labelfont=bf,up,textfont=it,up]{caption}
\usepackage[usenames,svgnames,dvipsnames]{xcolor}
\usepackage[a4paper,top=2.54cm,bottom=2.0cm,left=2.0cm,right=2.54cm]{geometry} % margens
\usepackage{amsmath}
\usepackage{booktabs} % cria tabelas em formato profissional
\usepackage[pdftex,plainpages=false,pdfpagelabels,pagebackref,colorlinks=true,citecolor=DarkGreen,
linkcolor=NavyBlue,urlcolor=DarkRed,filecolor=green,bookmarksopen=true]{hyperref} % links coloridos
\usepackage[all]{hypcap}                % soluciona o problema com o hyperref e capítulos
\usepackage[square,sort,nonamebreak,comma]{natbib}  % citação bibliográfica alpha
\fontsize{60}{62}\usefont{OT1}{cmr}{m}{n}{\selectfont}
\usepackage{upquote}                    % formata apóstrofes '
\usepackage{textcomp}

% Para formatar corretamente as URLs
\usepackage{url}
% -------------------------------------------------------------------- %
% Cabeçalhos similares ao TAOCP de Donald E. Knuth
\usepackage{fancyhdr}
\pagestyle{fancy}
\fancyhf{}
\renewcommand{\chaptermark}[1]{\markboth{\MakeUppercase{#1}}{}}
\renewcommand{\sectionmark}[1]{\markright{\MakeUppercase{#1}}{}}
\renewcommand{\headrulewidth}{0pt}

\frenchspacing                     % arruma o espaço: id est (i.e.) e exempli gratia (e.g.)
\urlstyle{same}                    % URL com o mesmo estilo do texto e no mono-spaced
\makeindex                         % para o índice remissivo
\raggedbottom                      % para no permitir espaços extras no texto
\fontsize{60}{62}\usefont{OT1}{cmr}{m}{n}{\selectfont}
\cleardoublepage
\normalsize

% -------------------------------------------------------------------- %
% Cores para formatação de código
\usepackage{color}
\definecolor{vermelho}{rgb}{0.6,0,0} % para strings
\definecolor{verde}{rgb}{0.25,0.5,0.35} % para comentários
\definecolor{roxo}{rgb}{0.5,0,0.35} % para palavras-chaves
\definecolor{azul}{rgb}{0.25,0.35,0.75} % para strings
\definecolor{cinza-claro}{gray}{0.95}
% -------------------------------------------------------------------- %
% Opções de listagem usados para o código fonte
% Ref: http://en.wikibooks.org/wiki/LaTeX/Packages/Listings
\usepackage{listings}           % para formatar código-fonte (ex. em Java)

\lstset{ %
language=Python,                      % seleciona a linguagem do código
basicstyle=\footnotesize\ttfamily,    % o tamanho da fonte usado no código
commentstyle=\color{verde}\bfseries,  % formatação de comentários
stringstyle=\color{azul},             % formatação de strings
upquote=true,
numbers=left,                   % onde colocar os números de linha
numberstyle=\tiny,  % o tamanho da fonte usada para a numeração das linhas
stepnumber=1,                   % o intervalo entre dois números de linhas. Se for 1, numera cada uma.
numbersep=5pt,                  % how far the line-numbers are from the code
showspaces=false,               % show spaces adding particular underscores
showstringspaces=false,         % underline spaces within strings
showtabs=false,                 % show tabs within strings adding particular underscores
keywordstyle=\color{roxo}\bfseries,
keywordstyle=[1]\color{roxo}\bfseries,
keywordstyle=[2]\color{verde}\bfseries,
frame=b,                   % adds a frame around the code
framerule=0.6pt,
tabsize=2,                      % sets default tabsize to 2 spaces
captionpos=t,                   % sets the caption-position to top
breaklines=true,                % sets automatic line breaking
breakatwhitespace=false,        % sets if automatic breaks should only happen at whitespace
escapeinside={\%*}{*)},         % if you want to add a comment within your code
backgroundcolor=\color[rgb]{1.0,1.0,1.0}, % choose the background color.
rulecolor=\color[rgb]{0.8,0.8,0.8},
extendedchars=true,
xleftmargin=10pt,
xrightmargin=10pt,
framexleftmargin=10pt,
framexrightmargin=10pt,
literate={â}{{\^{a}}}1  % para formatar corretamente os acentos do Português ao usar utf8
    {ê}{{\^{e}}}1
    {ô}{{\^{o}}}1
    {Â}{{\^{A}}}1
    {Ê}{{\^{E}}}1
    {Ô}{{\^{O}}}1
    {á}{{\'{a}}}1
    {é}{{\'{e}}}1
    {í}{{\'{i}}}1
    {ó}{{\'{o}}}1
    {ú}{{\'{u}}}1
    {Á}{{\'{A}}}1
    {É}{{\'{E}}}1
    {Í}{{\'{I}}}1
    {Ó}{{\'{O}}}1
    {Ú}{{\'{U}}}1
    {à}{{\`{a}}}1
    {À}{{\`{A}}}1
    {ã}{{\~{a}}}1
    {õ}{{\~{o}}}1
    {Ã}{{\~{A}}}1
    {Õ}{{\~{O}}}1
    {ç}{{\c{c}}}1
    {Ç}{{\c{C}}}1
    {ü}{{\"u}}1
    {Ü}{{\"U}}1
}

\renewcommand{\lstlistingname}{Listagem}
\renewcommand{\lstlistlistingname}{Lista de Listagens}

% \captionsetup[lstlisting]{singlelinecheck=false, labelfont={blue}, textfont={blue}}
\usepackage{caption}
\DeclareCaptionFont{white}{\color{white}}
\DeclareCaptionFormat{listing}{\colorbox[cmyk]{0.43, 0.35, 0.35,0.01}{\parbox{\textwidth}{\hspace{15pt}#1#2#3}}}
\captionsetup[lstlisting]{format=listing,labelfont=white,textfont=white, singlelinecheck=false, margin=0pt, font={bf,footnotesize}}

\title{Análise experimental de algoritmos usando Python}
\author{Patricia Mariana Ramos Marcolino\\
\texttt{\small \url{pmrmarcolino@hotmail.com}}
\vspace{1cm} \\
Eduardo Pinheiro Barbosa \\
\texttt{\small \url{eduardptu@hotmail.com}}
\vspace{1cm} \\
Faculdade de Computação \\
Universidade Federal de Uberlândia
}
\date{\today}

\begin{document}
\maketitle
% -------------------------------------------------------------------- %
% Listas de figuras, tabelas e códigos criadas automaticamente
\listoffigures
\listoftables
\lstlistoflistings
% -------------------------------------------------------------------- %

% -------------------------------------------------------------------- %
% Sumário
\tableofcontents
% cabeçalho para as páginas de todos os capítulos
\fancyhead[RE,LO]{\thesection}

%\singlespacing              % espaçamento simples
\setlength{\parskip}{0.15in} % espaçamento entre paragráfos

\chapter{Análise}
\subsection{Introdução}
O mergesort, é um algoritmo de ordenação que pode ser obtido especializando a formulação recursiva da divisão e onquista binária, onde divide todo o vetor usando recursividade até chegar ao caso base. após chegar ao caso base ele vem desempilhando comparando menores massas de dados. \\
Ideia básica:\\
Dividir:  dividir a lista em duas listas com cerca da metade do tamanho.\\
Conquistar:  dividir  cada  uma  das  d uas  sublistas  recursivamente  até  que  tenham  tamanho um.\\
Combinar:  fundir as duas sublistas de volta em uma lista ordenada.

\subsection{Desempenho do mergesort} 
Primeiramente vamos definir o que é melhor, médio e pior caso para o MergeSort. \\
Melhor Caso  – nunca é necessário trocar após comparações. \\
Médio Caso  – há necessidade de haver troca após comparações.\\ 
Pior Caso  – sempre é necessário trocar após comparações.\\
P ara o MergeSort não tem tanta importância se o vetor está no melhor, médio ou pior  caso, porque para qualquer que seja o caso ele sempre terá a complexidade de ordem  \[n*logn\]

\chapter{Resultados}
\section{Tabelas}

\input{../mergesort/tabelas/mergesortAleatorio.tex}

\begin{figure}[ht]
\centering \includegraphics[scale=0.8]{../mergesort/imagens/mergesortAleatorio0.png}
\caption{A análise do grafico para $2^{32}$ segue abaixo para mergesort de vetor aleatorio.\\
Tendo a função $T(n) = 5.3001*n*ln(n)-0.0075$ e para o $n =2^{32}$, $T(2^{32}) = 5.04916 * 10^{11}$}
\label{fig:mergesortAleatorio0}
\end{figure}

\begin{figure}[ht]
\centering \includegraphics[scale=0.8]{../mergesort/imagens/mergesortAleatorio1.png}
\caption{A análise do grafico para $2^{32}$ segue abaixo para mergesort de vetor aleatorio.\\
Tendo a função $T(n) = 0.1108*n*ln(n)+138.4156$ e para o $n =2^{32}$, $T(2^{32}) = 1.05554 * 10^{10}$}
\label{fig:mergesortAleatorio1}
\end{figure}


\input{../mergesort/tabelas/mergesortCrescente.tex}

\begin{figure}[ht]
\centering \includegraphics[scale=0.8]{../mergesort/imagens/mergesortCrescente0.png}
\caption{A análise do grafico para $2^{32}$ segue abaixo para mergesort de vetor crescente.\\
Tendo a função $T(n) = 5.2118\mathrm{e}-6*n*ln(n)-0.00719$ e para o $n =2^{32}$, $T(2^{32}) = 4.96504 * 10^{5}$}
\label{fig:mergesortCrescente0}
\end{figure}

\begin{figure}[ht]
\centering \includegraphics[scale=0.8]{../mergesort/imagens/mergesortCrescente1.png}
\caption{A análise do grafico para $2^{32}$ segue abaixo para mergesort de vetor crescente.\\
Tendo a função $T(n) = 0.11083*n*ln(n)-138.4156$ e para o $n =2^{32}$, $T(2^{32}) = 1.05583 * 10^{10}$}
\label{fig:mergesortCrescente1}
\end{figure}


\input{../mergesort/tabelas/mergesortDecrescente.tex}

\begin{figure}[ht]
\centering \includegraphics[scale=0.8]{../mergesort/imagens/mergesortDecrescente0.png}
\caption{A análise do grafico para $2^{32}$ segue abaixo para mergesort de vetor decrescente.\\
Tendo a função $T(n) = 5.2187\mathrm{e}-6*n*ln(n)-0.00706$ e para o $n =2^{32}$, $T(2^{32}) = 4.97162 * 10^{5}$}
\label{fig:mergesortDecrescente0}
\end{figure}

\begin{figure}[ht]
\centering \includegraphics[scale=0.8]{../mergesort/imagens/mergesortDecrescente1.png}
\caption{A análise numero de comparações do grafico para $2^{32}$ segue abaixo para mergesort de vetor crescente.\\
Tendo a função $T(n) = 0.11083*n*ln(n)-138.4156$ e para o $n =2^{32}$, $T(2^{32}) = 1.05583 * 10^{10}$}
\end{figure}


\input{../mergesort/tabelas/mergesortQuaseCresc10.tex}

\begin{figure}[ht]
\centering \includegraphics[scale=0.8]{../mergesort/imagens/mergesortQuaseCresc100.png}
\caption{A análise do grafico para $2^{32}$ segue abaixo para mergesort de vetor quase crescente 10\%.\\
Tendo a função $T(n) = 5.2357\mathrm{e}-6*n*ln(n)-0.0073$ e para o $n =2^{32}$, $T(2^{32}) = 4.98781 * 10^{5}$}
\label{fig:mergesortQuaseCresc100}
\end{figure}

\begin{figure}[ht]
\centering \includegraphics[scale=0.8]{../mergesort/imagens/mergesortQuaseCresc101.png}
\caption{A análise numero de comparações do grafico para $2^{32}$ segue abaixo para mergesort de vetor quase crescente 10\%.\\
Tendo a função $T(n) = 0.11083*n*ln(n)-138.4156$ e para o $n =2^{32}$, $T(2^{32}) = 1.05583 * 10^{10}$}
\label{fig:mergesortQuaseCresc101}
\end{figure}


\input{../mergesort/tabelas/mergesortQuaseCresc20.tex}

\begin{figure}[ht]
\centering \includegraphics[scale=0.8]{../mergesort/imagens/mergesortQuaseCresc200.png}
\caption{A análise do grafico para $2^{32}$ segue abaixo para mergesort de vetor quase crescente 20\%.\\
Tendo a função $T(n) = 5.1840\mathrm{e}-6*n*ln(n)-0.0066$ e para o $n =2^{32}$, $T(2^{32}) = 4.93856 * 10^{5}$}
\label{fig:mergesortQuaseCresc200}
\end{figure}

\begin{figure}[ht]
\centering \includegraphics[scale=0.8]{../mergesort/imagens/mergesortQuaseCresc201.png}
\caption{A análise numero de comparações do grafico para $2^{32}$ segue abaixo para mergesort de vetor quase crescente 20\%.\\
Tendo a função $T(n) = 0.11083*n*ln(n)-138.4156$ e para o $n =2^{32}$, $T(2^{32}) = 1.05583 * 10^{10}$}
\label{fig:mergesortQuaseCresc201}
\end{figure}

\clearpage
\input{../mergesort/tabelas/mergesortQuaseCresc30.tex}

\begin{figure}[ht]
\centering \includegraphics[scale=0.8]{../mergesort/imagens/mergesortQuaseCresc300.png}
\caption{A análise do grafico para $2^{32}$ segue abaixo para mergesort de vetor quase crescente 30\%.\\
Tendo a função $T(n) = 5.2132\mathrm{e}-6*n*ln(n)-0.007$ e para o $n =2^{32}$, $T(2^{32}) = 4.96638 * 10^{5}$}
\label{fig:mergesortQuaseCresc300}
\end{figure}

\begin{figure}[ht]
\centering \includegraphics[scale=0.8]{../mergesort/imagens/mergesortQuaseCresc301.png}
\caption{A análise numero de comparações do grafico para $2^{32}$ segue abaixo para mergesort de vetor quase crescente 30\%.\\
Tendo a função $T(n) = 0.11083*n*ln(n)-138.4156$ e para o $n =2^{32}$, $T(2^{32}) = 1.05583 * 10^{10}$}
\label{fig:mergesortQuaseCresc301}
\end{figure}


\input{../mergesort/tabelas/mergesortQuaseCresc40.tex}

\begin{figure}[ht]
\centering \includegraphics[scale=0.8]{../mergesort/imagens/mergesortQuaseCresc400.png}
\caption{A análise do grafico para $2^{32}$ segue abaixo para mergesort de vetor quase crescente 40\%.\\
Tendo a função $T(n) = 5.1550\mathrm{e}-6*n*ln(n)-0.0065$ e para o $n =2^{32}$, $T(2^{32}) = 4.91093 * 10^{5}$}
\label{fig:mergesortQuaseCresc400}
\end{figure}

\begin{figure}[ht]
\centering \includegraphics[scale=0.8]{../mergesort/imagens/mergesortQuaseCresc401.png}
\caption{A análise numero de comparações do grafico para $2^{32}$ segue abaixo para mergesort de vetor quase crescente 40\%.\\
Tendo a função $T(n) = 0.11083*n*ln(n)-138.4156$ e para o $n =2^{32}$, $T(2^{32}) = 1.05583 * 10^{10}$}
\label{fig:mergesortQuaseCresc401}
\end{figure}


\input{../mergesort/tabelas/mergesortQuaseCresc50.tex}

\begin{figure}[ht]
\centering \includegraphics[scale=0.8]{../mergesort/imagens/mergesortQuaseCresc500.png}
\caption{A análise do grafico para $2^{32}$ segue abaixo para mergesort de vetor quase crescente 50\%.\\
Tendo a função $T(n) = 5.2286\mathrm{e}-6*n*ln(n)-0.0071$ e para o $n =2^{32}$, $T(2^{32}) = 498105$}
\label{fig:mergesortQuaseCresc500}
\end{figure}

\begin{figure}[ht]
\centering \includegraphics[scale=0.8]{../mergesort/imagens/mergesortQuaseCresc501.png}
\caption{A análise numero de comparações do grafico para $2^{32}$ segue abaixo para mergesort de vetor quase crescente 50\%.\\
Tendo a função $T(n) = 0.11083*n*ln(n)-138.4156$ e para o $n =2^{32}$, $T(2^{32}) = 1.05583 * 10^{10}$}
\label{fig:mergesortQuaseCresc501}
\end{figure}


\input{../mergesort/tabelas/mergesortQuaseDecresc10.tex}

\begin{figure}[ht]
\centering \includegraphics[scale=0.8]{../mergesort/imagens/mergesortQuaseDecresc100.png}
\caption{A análise do grafico para $2^{32}$ segue abaixo para mergesort de vetor quase decrescente 10\%.\\
Tendo a função $T(n) = 5.1842\mathrm{e}-6*n*ln(n)-0.0068$ e para o $n =2^{32}$, $T(2^{32}) = 4.93875 * 10^{5}$}
\label{fig:mergesortQuaseDecresc100}
\end{figure}

\begin{figure}[ht]
\centering \includegraphics[scale=0.8]{../mergesort/imagens/mergesortQuaseDecresc101.png}
\caption{A análise numero de comparações do grafico para $2^{32}$ segue abaixo para mergesort de vetor quase decrescente 10\%.\\
Tendo a função $T(n) = 0.11083*n*ln(n)-138.4156$ e para o $n =2^{32}$, $T(2^{32}) = 1.05583 * 10^{10}$}
\label{fig:mergesortQuaseDecresc101}
\end{figure}


\input{../mergesort/tabelas/mergesortQuaseDecresc20.tex}

\begin{figure}[ht]
\centering \includegraphics[scale=0.8]{../mergesort/imagens/mergesortQuaseDecresc200.png}
\caption{A análise do grafico para $2^{32}$ segue abaixo para mergesort de vetor quase decrescente 20\%.\\
Tendo a função $T(n) = 5.4214\mathrm{e}-6*n*ln(n)-0.0077$ e para o $n =2^{32}$, $T(2^{32}) = 5.16472 * 10^{5}$}
\label{fig:mergesortQuaseDecresc200}
\end{figure}

\begin{figure}[ht]
\centering \includegraphics[scale=0.8]{../mergesort/imagens/mergesortQuaseDecresc201.png}
\caption{A análise numero de comparações do grafico para $2^{32}$ segue abaixo para mergesort de vetor quase decrescente 20\%.\\
Tendo a função $T(n) = 0.11083*n*ln(n)-138.4156$ e para o $n =2^{32}$, $T(2^{32}) = 1.05583 * 10^{10}$}
\label{fig:mergesortQuaseDecresc201}
\end{figure}

\clearpage
\input{../mergesort/tabelas/mergesortQuaseDecresc30.tex}

\begin{figure}[ht]
\centering \includegraphics[scale=0.8]{../mergesort/imagens/mergesortQuaseDecresc300.png}
\caption{A análise do grafico para $2^{32}$ segue abaixo para mergesort de vetor quase decrescente 30\%.\\
Tendo a função $T(n) = 5.1706\mathrm{e}-6*n*ln(n)-0.0067$ e para o $n =2^{32}$, $T(2^{32}) = 4.92579 * 10^{5}$}
\end{figure}

\begin{figure}[ht]
\centering \includegraphics[scale=0.8]{../mergesort/imagens/mergesortQuaseDecresc301.png}
\caption{A análise numero de comparações do grafico para $2^{32}$ segue abaixo para mergesort de vetor quase decrescente 30\%.\\
Tendo a função $T(n) = 0.11083*n*ln(n)-138.4156$ e para o $n =2^{32}$, $T(2^{32}) = 1.05583 * 10^{10}$}
\label{fig:mergesortQuaseDecresc301}
\end{figure}


\input{../mergesort/tabelas/mergesortQuaseDecresc40.tex}

\begin{figure}[ht]
\centering \includegraphics[scale=0.8]{../mergesort/imagens/mergesortQuaseDecresc400.png}
\caption{A análise do grafico para $2^{32}$ segue abaixo para mergesort de vetor quase decrescente 30\%.\\
Tendo a função $T(n) = 5.1621\mathrm{e}-6*n*ln(n)-0.0089$ e para o $n =2^{32}$, $T(2^{32}) = 4.91770 * 10^{5}$}
\label{fig:mergesortQuaseDecresc400}
\end{figure}

\begin{figure}[ht]
\centering \includegraphics[scale=0.8]{../mergesort/imagens/mergesortQuaseDecresc401.png}
\caption{A análise numero de comparações do grafico para $2^{32}$ segue abaixo para mergesort de vetor quase decrescente 40\%.\\
Tendo a função $T(n) = 0.11083*n*ln(n)-138.4156$ e para o $n =2^{32}$, $T(2^{32}) = 1.05583 * 10^{10}$}
\label{fig:mergesortQuaseDecresc401}
\end{figure}


\input{../mergesort/tabelas/mergesortQuaseDecresc50.tex}

\begin{figure}[ht]
\centering \includegraphics[scale=0.8]{../mergesort/imagens/mergesortQuaseDecresc500.png}
\caption{A análise do grafico para $2^{32}$ segue abaixo para mergesort de vetor quase decrescente 30\%.\\
Tendo a função $T(n) = 5.2936\mathrm{e}-6*n*ln(n)-0.0075$ e para o $n =2^{32}$, $T(2^{32}) = 5.04297 * 10^{5}$}
\label{fig:mergesortQuaseDecresc500}
\end{figure}

\begin{figure}[ht]
\centering \includegraphics[scale=0.8]{../mergesort/imagens/mergesortQuaseDecresc501.png}
\caption{A análise numero de comparações do grafico para $2^{32}$ segue abaixo para mergesort de vetor quase decrescente 50\%.\\
Tendo a função $T(n) = 0.11083*n*ln(n)-138.4156$ e para o $n =2^{32}$, $T(2^{32}) = 1.05583 * 10^{10}$}
\label{fig:mergesortQuaseDecresc501}
\end{figure}


\clearpage
\clearpage
\addcontentsline{toc}{part}{Apêndice}
\appendix

\chapter{Arquivo ../mergesort/mergesort.py \label{ap:mergesort}}
\lstinputlisting[caption={../mergesort/mergesort.py \label{arq:mergesort}}]{../mergesort/mergesort.py}

\chapter{Arquivo ../mergesort/ensaio.py \label{ap:mergesortensaio}}
\lstinputlisting[caption={../mergesort/ensaio.py \label{arq:mergesortensaio}}]{../mergesort/ensaio.py}

\end{document}
