\documentclass[12pt,a4paper,twoside]{report}
% -------------------------------------------------------------------- %
% Pacotes

\usepackage[utf8]{inputenc}
\usepackage[T1]{fontenc}
\usepackage[brazil]{babel}
\usepackage[fixlanguage]{babelbib}
\usepackage[pdftex]{graphicx}      % usamos arquivos pdf/png como figuras
\usepackage{setspace}              % espaçamento flexvel
\usepackage{indentfirst}           % indentação do primeiro parágrafo
\usepackage{makeidx}               % índice remissivo
\usepackage[nottoc]{tocbibind}     % acrescentamos a bibliografia/indice/conteudo no Table of Contents
\usepackage{courier}               % usa o Adobe Courier no lugar de Computer Modern Typewriter
\usepackage{type1cm}               % fontes realmente escaláveis
\usepackage{titletoc}
\usepackage{ucs}
\usepackage[font=small,format=plain,labelfont=bf,up,textfont=it,up]{caption}
\usepackage[usenames,svgnames,dvipsnames]{xcolor}
\usepackage[a4paper,top=2.54cm,bottom=2.0cm,left=2.0cm,right=2.54cm]{geometry} % margens
\usepackage{amsmath}
\usepackage{booktabs} % cria tabelas em formato profissional
\usepackage[pdftex,plainpages=false,pdfpagelabels,pagebackref,colorlinks=true,citecolor=DarkGreen,
linkcolor=NavyBlue,urlcolor=DarkRed,filecolor=green,bookmarksopen=true]{hyperref} % links coloridos
\usepackage[all]{hypcap}                % soluciona o problema com o hyperref e capítulos
\usepackage[square,sort,nonamebreak,comma]{natbib}  % citação bibliográfica alpha
\fontsize{60}{62}\usefont{OT1}{cmr}{m}{n}{\selectfont}
\usepackage{upquote}                    % formata apóstrofes '
\usepackage{textcomp}

% Para formatar corretamente as URLs
\usepackage{url}
% -------------------------------------------------------------------- %
% Cabeçalhos similares ao TAOCP de Donald E. Knuth
\usepackage{fancyhdr}
\pagestyle{fancy}
\fancyhf{}
\renewcommand{\chaptermark}[1]{\markboth{\MakeUppercase{#1}}{}}
\renewcommand{\sectionmark}[1]{\markright{\MakeUppercase{#1}}{}}
\renewcommand{\headrulewidth}{0pt}

\frenchspacing                     % arruma o espaço: id est (i.e.) e exempli gratia (e.g.)
\urlstyle{same}                    % URL com o mesmo estilo do texto e no mono-spaced
\makeindex                         % para o índice remissivo
\raggedbottom                      % para no permitir espaços extras no texto
\fontsize{60}{62}\usefont{OT1}{cmr}{m}{n}{\selectfont}
\cleardoublepage
\normalsize

% -------------------------------------------------------------------- %
% Cores para formatação de código
\usepackage{color}
\definecolor{vermelho}{rgb}{0.6,0,0} % para strings
\definecolor{verde}{rgb}{0.25,0.5,0.35} % para comentários
\definecolor{roxo}{rgb}{0.5,0,0.35} % para palavras-chaves
\definecolor{azul}{rgb}{0.25,0.35,0.75} % para strings
\definecolor{cinza-claro}{gray}{0.95}
% -------------------------------------------------------------------- %
% Opções de listagem usados para o código fonte
% Ref: http://en.wikibooks.org/wiki/LaTeX/Packages/Listings
\usepackage{listings}           % para formatar código-fonte (ex. em Java)

\lstset{ %
language=Python,                      % seleciona a linguagem do código
basicstyle=\footnotesize\ttfamily,    % o tamanho da fonte usado no código
commentstyle=\color{verde}\bfseries,  % formatação de comentários
stringstyle=\color{azul},             % formatação de strings
upquote=true,
numbers=left,                   % onde colocar os números de linha
numberstyle=\tiny,  % o tamanho da fonte usada para a numeração das linhas
stepnumber=1,                   % o intervalo entre dois números de linhas. Se for 1, numera cada uma.
numbersep=5pt,                  % how far the line-numbers are from the code
showspaces=false,               % show spaces adding particular underscores
showstringspaces=false,         % underline spaces within strings
showtabs=false,                 % show tabs within strings adding particular underscores
keywordstyle=\color{roxo}\bfseries,
keywordstyle=[1]\color{roxo}\bfseries,
keywordstyle=[2]\color{verde}\bfseries,
frame=b,                   % adds a frame around the code
framerule=0.6pt,
tabsize=2,                      % sets default tabsize to 2 spaces
captionpos=t,                   % sets the caption-position to top
breaklines=true,                % sets automatic line breaking
breakatwhitespace=false,        % sets if automatic breaks should only happen at whitespace
escapeinside={\%*}{*)},         % if you want to add a comment within your code
backgroundcolor=\color[rgb]{1.0,1.0,1.0}, % choose the background color.
rulecolor=\color[rgb]{0.8,0.8,0.8},
extendedchars=true,
xleftmargin=10pt,
xrightmargin=10pt,
framexleftmargin=10pt,
framexrightmargin=10pt,
literate={â}{{\^{a}}}1  % para formatar corretamente os acentos do Português ao usar utf8
    {ê}{{\^{e}}}1
    {ô}{{\^{o}}}1
    {Â}{{\^{A}}}1
    {Ê}{{\^{E}}}1
    {Ô}{{\^{O}}}1
    {á}{{\'{a}}}1
    {é}{{\'{e}}}1
    {í}{{\'{i}}}1
    {ó}{{\'{o}}}1
    {ú}{{\'{u}}}1
    {Á}{{\'{A}}}1
    {É}{{\'{E}}}1
    {Í}{{\'{I}}}1
    {Ó}{{\'{O}}}1
    {Ú}{{\'{U}}}1
    {à}{{\`{a}}}1
    {À}{{\`{A}}}1
    {ã}{{\~{a}}}1
    {õ}{{\~{o}}}1
    {Ã}{{\~{A}}}1
    {Õ}{{\~{O}}}1
    {ç}{{\c{c}}}1
    {Ç}{{\c{C}}}1
    {ü}{{\"u}}1
    {Ü}{{\"U}}1
}

\renewcommand{\lstlistingname}{Listagem}
\renewcommand{\lstlistlistingname}{Lista de Listagens}

% \captionsetup[lstlisting]{singlelinecheck=false, labelfont={blue}, textfont={blue}}
\usepackage{caption}
\DeclareCaptionFont{white}{\color{white}}
\DeclareCaptionFormat{listing}{\colorbox[cmyk]{0.43, 0.35, 0.35,0.01}{\parbox{\textwidth}{\hspace{15pt}#1#2#3}}}
\captionsetup[lstlisting]{format=listing,labelfont=white,textfont=white, singlelinecheck=false, margin=0pt, font={bf,footnotesize}}

\title{Análise experimental de algoritmos usando Python}
\author{Patrícia Mariana Ramos Marcolino \\
\texttt{\small \url{pmrmarcolino@hotmail.com}}
\vspace{1cm} \\
Eduardo Pinheiro Barbosa\\
\texttt{\small \url{eduardptu@hotmail.com}}
\vspace{1cm} \\
Faculdade de Computação \\
Universidade Federal de Uberlândia
}
\date{\today}

\begin{document}
\maketitle
% -------------------------------------------------------------------- %
% Listas de figuras, tabelas e códigos criadas automaticamente
\listoffigures
\listoftables
\lstlistoflistings
% -------------------------------------------------------------------- %

% -------------------------------------------------------------------- %
% Sumário
\tableofcontents
% cabeçalho para as páginas de todos os capítulos
\fancyhead[RE,LO]{\thesection}

%\singlespacing              % espaçamento simples
\setlength{\parskip}{0.15in} % espaçamento entre paragráfos

\chapter{Análise}

O algoritmo {\it Bucket Sort} possui tempo linear, desde que
os valores a serem ordenados sejam distribuídos
uniformemente sobre o intervalo $[0, 1)$.\\
O {\it Bucket Sort} divide o intervalo
[0, 1) em n sub-intervalos iguais, denominados buckets
(baldes), e então distribui os n números reais nos n
buckets. Como a entrada é composta por dados
distribuídos uniformemente, espera-se que cada balde
possua, ao final deste processo, um número equivalente
de elementos (usualmente 1).\\
Para obter o resultado, basta ordenar os elementos em
cada bucket e então apresentá-los em ordem.\\
Ordena n números uniformemente
distribuídos na faixa [0-1) em tempo médio O(n).
\\Sabendo que o bucket sort não trabalha com o método de comparação, todos os graficos respectivos a essa informação são constantes.
\chapter{Resultados}
\section{Tabelas}

\input{../bucketsort/tabelas/bucketsortAleatorio.tex}

\begin{figure}[ht]
\centering \includegraphics[scale=0.8]{../bucketsort/imagens/bucketsortAleatorio0.png}
\caption{A análise do grafico para $2^{32}$ segue abaixo para bucketsort}

Tendo a função $T(n) = 2.839\mathrm{e}-5*n-0.00191$ e para o $n =2^{32}$, $T(2^{32}) = 121934.11962344$ 
\label{fig:bucketsortAleatorio0}
\end{figure}

\begin{figure}[ht]
\centering \includegraphics[scale=0.8]{../bucketsort/imagens/bucketsortAleatorio1.png}
\caption{A análise do grafico para $2^{32}$ segue abaixo para bucketsort}

Tendo a função $T(n) = 1 $ 
\label{fig:bucketsortAleatorio1}
\end{figure}


\input{../bucketsort/tabelas/bucketsortCrescente.tex}

\begin{figure}[ht]
\centering \includegraphics[scale=0.8]{../bucketsort/imagens/bucketsortCrescente0.png}
\caption{ A análise do grafico para $2^{32}$ segue abaixo para bucketsort}

Tendo a função $T(n) = 1.092\mathrm{e}-5*n-9.822\mathrm{e}-5$ e para o $n =2^{32}$, $T(2^{32}) = 46901.0427741$ 
\label{fig:bucketsortCrescente0}
\end{figure}

\input{../bucketsort/tabelas/bucketsortDecrescente.tex}

\begin{figure}[ht]
\centering \includegraphics[scale=0.8]{../bucketsort/imagens/bucketsortDecrescente0.png}
\caption{A análise do grafico para $2^{32}$ segue abaixo para bucketsort}

Tendo a função $T(n) = 4.955\mathrm{e}-5*n-0.004879$ e para o $n =2^{32}$, $T(2^{32}) = 212815.6246378$ 
\label{fig:bucketsortDecrescente0}
\end{figure}

\input{../bucketsort/tabelas/bucketsortQuaseCresc10.tex}

\begin{figure}[ht]
\centering \includegraphics[scale=0.8]{../bucketsort/imagens/bucketsortQuaseCresc100.png}
\caption{A análise do grafico para $2^{32}$ segue abaixo para bucketsort}

Tendo a função $T(n) = 1.146\mathrm{e}-5*n+0.0002222$ e para o $n =2^{32}$, $T(2^{32}) = 49220.32543436$ 
\label{fig:bucketsortQuaseCresc100}
\end{figure}


\input{../bucketsort/tabelas/bucketsortQuaseCresc20.tex}

\begin{figure}[ht]
\centering \includegraphics[scale=0.8]{../bucketsort/imagens/bucketsortQuaseCresc200.png}
\caption{A análise do grafico para $2^{32}$ segue abaixo para bucketsort}

Tendo a função $T(n) = 1.287\mathrm{e}-5*n-0.0001216$ e para o $n =2^{32}$, $T(2^{32}) = 55276.22897792$ 
\label{fig:bucketsortQuaseCresc200}
\end{figure}

\clearpage
\input{../bucketsort/tabelas/bucketsortQuaseCresc30.tex}

\begin{figure}[ht]
\centering \includegraphics[scale=0.8]{../bucketsort/imagens/bucketsortQuaseCresc300.png}
\caption{A análise do grafico para $2^{32}$ segue abaixo para bucketsort}

Tendo a função $T(n) = 1.373\mathrm{e}-5*n-0.0001417$ e para o $n =2^{32}$, $T(2^{32}) = 58969.90083238$ 
\label{fig:bucketsortQuaseCresc300}
\end{figure}


\input{../bucketsort/tabelas/bucketsortQuaseCresc40.tex}

\begin{figure}[ht]
\centering \includegraphics[scale=0.8]{../bucketsort/imagens/bucketsortQuaseCresc400.png}
\caption{A análise do grafico para $2^{32}$ segue abaixo para bucketsort}

Tendo a função $T(n) = 1.555\mathrm{e}-5*n-0.0006026$ e para o $n =2^{32}$, $T(2^{32}) = 66786.7408502$ 
\label{fig:bucketsortQuaseCresc400}
\end{figure}

\input{../bucketsort/tabelas/bucketsortQuaseCresc50.tex}

\begin{figure}[ht]
\centering \includegraphics[scale=0.8]{../bucketsort/imagens/bucketsortQuaseCresc500.png}
\caption{A análise do grafico para $2^{32}$ segue abaixo para bucketsort}

Tendo a função $T(n) = 1.562\mathrm{e}-5*n-0.0004459$ e para o $n =2^{32}$, $T(2^{32}) =  67087.38871762$ 
\label{fig:bucketsortQuaseCresc500}
\end{figure}

\input{../bucketsort/tabelas/bucketsortQuaseDecresc10.tex}

\begin{figure}[ht]
\centering \includegraphics[scale=0.8]{../bucketsort/imagens/bucketsortQuaseDecresc100.png}
\caption{A análise do grafico para $2^{32}$ segue abaixo para bucketsort}

Tendo a função $T(n) = 4.956\mathrm{e}-5*n-0.005513$ e para o $n =2^{32}$, $T(2^{32}) = 212858.57367676 $ 
\label{fig:bucketsortQuaseDecresc100}
\end{figure}


\input{../bucketsort/tabelas/bucketsortQuaseDecresc20.tex}

\begin{figure}[ht]
\centering \includegraphics[scale=0.8]{../bucketsort/imagens/bucketsortQuaseDecresc200.png}
\caption{EA análise do grafico para $2^{32}$ segue abaixo para bucketsort}

Tendo a função $T(n) = 5.032\mathrm{e}-5*n-0.005975$ e para o $n =2^{32}$, $T(2^{32}) = 216122.74835972$ 
\label{fig:bucketsortQuaseDecresc200}
\end{figure}

\clearpage
\input{../bucketsort/tabelas/bucketsortQuaseDecresc30.tex}

\begin{figure}[ht]
\centering \includegraphics[scale=0.8]{../bucketsort/imagens/bucketsortQuaseDecresc300.png}
\caption{A análise do grafico para $2^{32}$ segue abaixo para bucketsort}

Tendo a função $T(n) = 4.468\mathrm{e}-5*n-0.004343$ e para o $n =2^{32}$, $T(2^{32}) = 191899.13444228$ 
\label{fig:bucketsortQuaseDecresc300}
\end{figure}

\input{../bucketsort/tabelas/bucketsortQuaseDecresc40.tex}

\begin{figure}[ht]
\centering \includegraphics[scale=0.8]{../bucketsort/imagens/bucketsortQuaseDecresc400.png}
\caption{A análise do grafico para $2^{32}$ segue abaixo para bucketsort}

Tendo a função $T(n) = 4.707\mathrm{e}-5*n-0.005511$ e para o $n =2^{32}$, $T(2^{32}) = 202164.10511172$ 
\label{fig:bucketsortQuaseDecresc400}
\end{figure}


\input{../bucketsort/tabelas/bucketsortQuaseDecresc50.tex}

\begin{figure}[ht]
\centering \includegraphics[scale=0.8]{../bucketsort/imagens/bucketsortQuaseDecresc500.png}
\caption{EA análise do grafico para $2^{32}$ segue abaixo para bucketsort}

Tendo a função $T(n) = 4.275\mathrm{e}-5*n-0.003819$ e para o $n =2^{32}$, $T(2^{32}) = 183609.848085$ 
\end{figure}


\clearpage
\clearpage
\addcontentsline{toc}{part}{Apêndice}
\appendix

\chapter{Arquivo ../bucketsort/bucketsort.py \label{ap:bucketsort}}
\lstinputlisting[caption={../bucketsort/bucketsort.py \label{arq:bucketsort}}]{../bucketsort/bucketsort.py}

\chapter{Arquivo ../bucketsort/ensaio.py \label{ap:bucketsortensaio}}
\lstinputlisting[caption={../bucketsort/ensaio.py \label{arq:bucketsortensaio}}]{../bucketsort/ensaio.py}

\end{document}
